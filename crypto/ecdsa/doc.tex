%TEX program=xelatex
\documentclass[a4paper,10pt]{article}

\usepackage{ctex}
\usepackage{adjustbox}
\usepackage{amsmath}
\usepackage[linesnumbered,ruled,vlined]{algorithm2e}
\usepackage[backend=biber,style=alphabetic]{biblatex}
\usepackage[margin=16mm]{geometry}
\usepackage{hyperref}
\usepackage{indentfirst}
\usepackage{listings}
\usepackage{tcolorbox}
\usepackage{xcolor}

\title{ECDSA签名算法}
\author{Sammy}
\date{2018-04-10}

\addbibresource{REFERENCES.bib}
\hypersetup{
    colorlinks=true,
		citecolor=blue,
    linkcolor=blue,
    filecolor=magenta,      
    urlcolor=blue,
}
\urlstyle{same}

\lstset{
  basicstyle=\ttfamily,           % the size of the fonts that are used for the mkeyword
  numbers=left,
  tabsize=2,
  breaklines=true,
  keywordstyle=\color{magenta},          % keyword style
  commentstyle=\color{red!70!green},       % comment style
  stringstyle=\color{blue},          % string literal style
  frame=trBL,     
  numberstyle=\footnotesize,
%  morecomment=[l]{\#},
  morecomment=[s]{/*}{*/},
  escapeinside={(*@}{@*)},
  showspaces=false,
}

% style for keywords
\newcommand{\primaryKW}[1]{\textcolor{blue}{\lstinline!#1!}}
\newcommand{\successKW}[1]{\textcolor{green}{\lstinline!#1!}}
\newcommand{\infoKW}[1]{\textcolor{cyan}{\lstinline!#1!}}
\newcommand{\warningKW}[1]{\textcolor{orange}{\lstinline!#1!}}
\newcommand{\dangerKW}[1]{\textcolor{red}{\lstinline!#1!}}

% badge
\newcommand{\badge}[2]{{\scriptsize\tcbox[arc=1.5mm,colback=#1,colframe=#1,coltext=white,height=4mm,on line,valign=center,left=-1mm,right=-1mm]{\textbf{#2}}}}

\begin{document}
\maketitle

\section{公开参数}
ECDSA基于素数域的椭圆曲线,相关参数如\tablename~\ref{tab-param}所示  
\begin{table}[!htbp]
  \centering
  \caption{ECDSA涉及的公开参数}\label{tab-param}
  \begin{tabular}{rl}
    \hline
    符号 &含义 \\
    \hline
    \(O\) &无穷远点 \\
    \(q\) &一个大素数 \\ 
    \(b\) &椭圆曲线方程\(y^2\equiv x^3-3x+b\)中的常数 \\
    \(G\) &基点\((x_g,y_g)\),选定的曲线上的一个点 \\
    \(n\) &\(G\)的阶,满足\(nG=O\) \\
    \(H\) &哈希算法 \\
    \hline
  \end{tabular}
\end{table}

\section{具体算法}
组成算法的3个关键流程分别如算法~\ref{algo-keygen}、\ref{algo-sign}、\ref{algo-verify}所示

\begin{algorithm}
  \DontPrintSemicolon
  \SetKwInOut{Input}{输入}
  \SetKwInOut{Output}{输出}

  \Output{私钥\(d\),公钥\(Q\)}
  \BlankLine

  \(d\in_R [1,n-1]\) \tcc*[r]{随机生成一个整数\(d\)}
  \(Q\leftarrow d\cdot G\)\;
  \caption{ECDSA密钥生成}\label{algo-keygen}
\end{algorithm}

\begin{adjustbox}{left=0.45\linewidth,minipage=0.45\linewidth,valign=T}
  \begin{algorithm}[H]
    \DontPrintSemicolon
    \SetKwInOut{Input}{输入}
    \SetKwInOut{Output}{输出}

    \Input{私钥\(d\),消息\(m\)}
    \Output{签名\((r,s)\)}
    \BlankLine

    \(r\leftarrow 0\)\;
    \While{\(r=0\)}{
      \(k\in_R [1,n-1]\)\;
      \(P=(x,y)=k\cdot G\)\;
      \(r\equiv x\pmod{n}\)\;
    }
    \(e\leftarrow H(m)\)\;
    \(s\leftarrow 0\)\;
    \While{\(s=0\)}{
      \(s\leftarrow k^{-1}(e+d\cdot r)\pmod{n}\)\;
    }

    \caption{ECDSA签名}\label{algo-sign}
  \end{algorithm}
\end{adjustbox}  
\begin{adjustbox}{left=0.45\linewidth,minipage=0.45\linewidth,valign=T}
  \begin{algorithm}[H]
    \DontPrintSemicolon
    \SetKwInOut{Input}{输入}
    \SetKwInOut{Output}{输出}

    \Input{公钥\(Q\),消息\(m\),签名\((r,s)\)}
    \Output{\(Y\)表示签名合法,\(N\)表示签名非法}
    \BlankLine

    \If{\(r\not\in[1,n-1]\)或\(s\not\in[1,n-1]\)}{
      \Return{\(N\)}\;
    }
    \(e\leftarrow H(m)\)\;
    \(w\leftarrow s^{-1}\pmod{n}\)\;
    \(A\leftarrow (x_1,y_1)=ewG+rwQ\)\;
    \If{\(A=O\)}{
      \Return{\(N\)};
    }
    \(v\leftarrow x_1\pmod{n}\)\;
    \lIf{\(v=r\)}{ \Return{\(Y\)}}
    \lElse{ \Return{\(N\)} }

    \caption{ECDSA验签}\label{algo-verify}
  \end{algorithm}
\end{adjustbox}  

\section{正确性证明}
\[
  \begin{aligned}  
    &s \equiv k^{-1}(e+d\cdot r)\pmod{n} \\
    \Rightarrow &k \equiv s^{-1}(e+d\cdot r)\pmod{n} \\
    \Rightarrow &k \equiv (s^{-1}\cdot e+s^{-1}\cdot d\cdot r)\pmod{n} \\
    \Rightarrow &k \equiv (e\cdot w+r\cdot w\cdot d)\pmod{n} \\
    \Rightarrow &kG \equiv (e\cdot w+r\cdot w\cdot d)\cdot G\pmod{n} \\
    \Rightarrow &kG \equiv e\cdot w\cdot G +r\cdot w\cdot d\cdot G \pmod{n} \\
    \Rightarrow &kG \equiv e\cdot w\cdot G +r\cdot w\cdot Q \pmod{n} \\
    \Rightarrow &P=A \\
    \Rightarrow &r\equiv x\equiv x_1\equiv v\pmod{n}
  \end{aligned}  
\]

\section{Go语言的ECDSA库}
Go语言crypto/elliptic包提供了FIPS 186-3标准规定的4种曲线P224、P256、P384和P512,我们根据自己的安全强度需求直接利用这些曲线就行。其中,紧接P的数字表示以比特为单位衡量的安全强度,越大越强。

相应的演示程序如\lstinline!ecdsa_test.go!的\lstinline!TestECDSA!函数。

\end{document}
